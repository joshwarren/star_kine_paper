%\documentclass[a4paper,11pt]{article}
\documentclass[fleqn,usenatbib,useAMS]{mnras}
%\usepackage{amsmath}
\usepackage{graphicx} % For figures
%\usepackage{subcaption} % for multiple plots in one figures - allows caption and referencing plots and figures

%\usepackage{amssymb}
%\usepackage{verbatim}
%\usepackage{float} % Not installed
%\usepackage{multirow} % Not installed
%\usepackage[section]{placeins} % Not installed - think is for floatbarrier command
%\usepackage{cite}
%\usepackage{natbib}
%\usepackage{newtxtext,newtxmath} % Times font as in MNRAS
\usepackage{mathptmx}
\usepackage{txfonts}


%\numberwithin{equation}{section}

%\newcommand{\nocontentsline}[3]{}
%\newcommand{\tocless}[2]{\bgroup\let\addcontentsline=\nocontentsline#1{#2}\egroup}
%\allowdisplaybreaks
%\newcommand{\HRule}{\rule{\linewidth}{0.5mm}}



\title{Low-powered Radio Galaxies: Spatially Resolved Kinematics}
\author[J. Warren et al.]{
Joshua Warren,$^{1}$\thanks{Contact e-mail: \href{mailto:joshua.warren@physics.ox.ac.uk}{joshua.warren@physics.ox.ac.uk}}
Martin Bureau,$^{1}$
Bernd Hasemann,$^{2}$
Isabella Prandoni,$^{3}$ \newauthor
Francesco Santoro,$^{3}$
Robert Laing,$^{3}$
Paola Parma,$^{3}$
Hans de Ruiter$^{3}$
and Arturo Mignano$^{3}$
\\
$^{1}$Sub-department of Astrophysics, Department of Physics, University of Oxford, Denys Wilkinson Building, Keble Road, Oxford OX1 3RH, UK\\
$^{2}$European Southern Observatory, Karl-Schwarzschild-Str. 2, 85748 Garching b. München, Germany\\
$^{3}$INAF - Istituto di Radioastronomia, Via P. Gobetti 101, 40129 Bologna, Italy}

\begin{document}
\maketitle

%\date{Oxford, \today} % date is given automatically at top of page in mnras formatting

\begin{abstract}

\end{abstract}



\section{Introduction}
	\label{sec:intro}
	There are known to be strong scaling relations between the mass of the central supermassive black-hole (SMBH) and the global properties in a given galaxy, such as velocity dispersion, the $M-\sigma$ relation \citep{Ferrarese2000, Gebhardt2000, Graham2011}; Sersic index \citep{Graham2007, Savorgnan2013}; luminosity \citep{Laor2001, McLure2001, Lauer2007, Graham2012}; etc. AGN feedback is understood to be the mechanism by which the SMBH can expand its sphere of influence to galactic scales to affect these relationship. 

	This is generally understood to be a feedback process, where large amounts of gas within the galaxy cools, falls on the SMBH. Here the black-hole imparts energy to the gas, causing outflows (or jets) and/or thermal emission. We observe this as an Active Galactic Nuclei (AGN) which can have distinctive emission across the electromagnetic spectrum. The exact mechanisms of imparting energy still remains an open question, with much theoretical and observational work ongoing. The outflows can then effect the outer parts of the galaxy, by methods such as fountain effects \citep{}, thermal heating \citep{DeYoung2010} and shocks created by the outflows \citep{}.

	One identifier of AGNs is a radio signature consistent with Synchrotron emission from the outflow i.e a radio galaxy (RG). These are typically categorized into one of two classes: Fanaroff-Riley (FR) I or FR II \citep{Fanaroff1974}. Of these we focus on the former since these are by far the most common mode in the local universe (possibly as far as $z \sim 1$ \citep{Rigby2008}), and so to discuss radio AGN feedback is to consider feedback from FR I sources \citep{DeYoung2010}. Indeed, while there is broad consensus \citep{Heckman1986, Baum1992} that the brightest RGs (mostly FR IIs) are caused when a massive galaxy merges with a gas rich galaxy (a wet merger) giving a plentiful fuel reservoir to the AGN \citep{Baum1992}, there is much more debate when it comes to low-powered ($P_\mathrm{1.4 GHz} \lesssim 10^{24.5} \, \mathrm{W Hz^{-1}}$) RGs (mostly FR I). Suggestions fall into two categories: an extrapolation of radio-loud galaxies (i.e. a merger with either small gas reservoirs or low efficiency accretion onto the SMBH) or a highly efficient accretion of gas from secular origins (such as Bondi accretion of the hot X-ray component of the Inter-Stellar Medium (ISM) \citep{Allen2006} or from existing cold gas reservoirs \citep{Prandoni2010}). This project aims to add to this discussion using Integral Field Unit (IFU) observations in the visible band of nearby radio galaxies and comparing this with Atlas-3D sample \citep{Cappellari2011} as a control sample.

% More to add to intro - e.g. summery of paper 

\section{Sample Description}
	\label{sec:samp}
	The sample was first set out in \citet{Prandoni2010}. The Parkes 2.7 GHz survey was used as the parent sample to identify radio galaxies. This imposed a declination range of $-14\deg < \delta < -40\deg$ and a flux density limit of 0.25 Jy in the 2.7 GHz band. The additional criteria was an associated early type galaxy (ETG) host which is at $z<0.03$. This left 11 sources, which hereafter will be referred to as the Southern Sample. All display an FRI morphology with low to intermediate radio power.

	Of this sample, 10 were observed with the APEX single dish telescope for CO, with the remaining host (NGC 1316) already observed in \citet{Horellou2001}. All were detected and results are given in \citet{Prandoni2010}. These detections have since been followed up with ALMA (not yet released). 


\section{Observing strategy and Data Reduction}
	\label{sec:obs}
	\subsection{The VIsable Multi-Object Spectrograph (VIMOS)}
		\label{subsec:VIMOS}
		The sample was observed with the VIsable Multi-Object Spectrograph (VIMOS), mounted on UT3 on the VLT in Paranal \citep{LeFevre2003}, using the new (at the time) HR Blue grism. All observations were taken with a spacial resolution of 0.67". Each object was imaged with a total integration time of $\sim 100$ mins equally spread over three observing blocks. Each block contained all of the necessary calibration images (3 flat fields and 1 He and Ne arc lamp image for wavelength calibration), as well as two science pointings. In addition, VIMOS provides 5 bias images per night. Flux calibrations was done using public Standards provided by ESO of Feige 110.

		VIMOS has several well known, though not well understood technical issues. These include several low transmission (bad) fibers, strong flexure and large differences in sensitivity across its 4 separate detectors, known as quadrants. These are addressed by a specialist data reduction pipeline as described in Section \ref{subsec:reduct}. 

	\subsection{Data Reduction}
		\label{subsec:reduct}
		The data reduction pipeline was produced using Py3D, a suite of programs based on those developed for Califa DR1 \citep{Sanchez2011, Husemann2013} and later updated for VIMOS by \citet{Husemann2014}. This pipeline accounts for many of the known issues with VIMOS such as bad fibers and strong flexure. A brief outline is the additional steps to accounts for these is given below, while detailed descriptions can be found in \citet{Sanchez2011} for the standard reduction steps (bias subtraction, flat-fielding and wavelength and flux calibrations) and \citep{Husemann2014} for the VIMOS specific modifications. 
		\begin{itemize}
		\item The known bad pixels and flexure offsets are included when completing the wavelength calibration. In testing, this was found to be robust except for the blue end of the spectrum ($\lambda < 4300 \AA$) when using the blue HR grism \citep{Husemann2014}. Indeed, we found that the spectrum below 4200\AA was strongly suppressed and was therefore discarded.
		\item Flexure also gives an offset to flat-fielding and while this known to be only up to $\pm0.5$pixels, this is also taken into account.
		\item The dense packing of the fibers onto the CCD means that cross-talk is occurs and is extracted using \citep{Horne1986}. 
		\end{itemize}
		All of the above steps are detailed in Section 2.3 in \citet{Husemann2014}.

		%After inspecting the reduced data cubes, it was noted that most of the outer fibres had no (or very low) transmission. Given the observations were all centally located on the field of view, the outer 2 rows of fibres on each edge were able to be discarded to leave a final cube of the central 36 x 36 fibres or a field of view of 24.1 x 24.1 arcsecs. 

		Following on from this, it was noted that the cubes where still not fully corrected. A fringe-like pattern was still observable in the spectral direction and quadrants were not calibrated to each other. These were improved by implementing a python version of the ad-hoc corrections given in \citet{Lagerholm2012}. This involves re-normalizing the quadrants by minimizing the difference of the integrated spectra in neighboring fibers (Q2 was held constant) followed by the removal of a fringe-like pattern, by dividing out a smoothed median spectrum from the eight surrounding fibers of any given fiber, over a scale of 150 pixels. 

		The variance spectra is propagated throughout the data reduction pipeline to be used a noise input in the analysis (\S \ref{subsec:analysis})

		% Need to show some examples. 
	\subsection{Data Analysis}
		\label{subsec:analysis}
		The cube was then spatially binned using Voronoi Binning technique \citep{Cappellari2003} to a Signal to Noise ratio (S/N), defined to be the ratio of the medians of the spatially integrated cube and the spatially integrated noise spectrum, of 30. Below about the calcium doublet, Ca4227, the observed spectrum tended to become unresponsive and so a hard cut lower cut was applied at 4200 \AA. When calculating total flux or any parameter dependent on total flux, only the wavelength range common to all bins was used (the quadrants have slightly different wavelength ranges), otherwise this was computed individually for each bin in order to preserve as much as information as possible.

		Following this the cubes were analyzed using the pPXF package \citep{Cappellari2004} to find a best fit spectrum by stacking the Miles empirical stellar spectra \citep{Sanchez-Blazquez2006} convolved with a line-of-sight velocity distribution (LOSVD) given by a Gaussian-like distribution, allowed to vary up to the 4th Gauss-Hermite moment (i.e. the LOSVD could be characterized by 4 parameters: recessional velocity, $v$; line-of-sight velocity dispersion, $\sigma$ and the 3rd and 4th Gauss-Hermite moments, $h_3$ and $h_4$). The fit was done within a simple Monty Carlo (MC) method to allow accurate estimation of the uncertainties in these values, by adding noise to the spectra and investigating the effect on the LOSVD. The [OIII] doublet, H$_\mathrm{\beta}$, H$_\mathrm{\gamma}$, H$_\mathrm{\delta}$ and [NI] emission lines were also simultaneously fitted with the same method, using a Gaussian spectrum for the template, each with their own free LOSVD. A fourth order additive Legendre polynomial was allowed as a free parameter to correct for the continuum shape. A short MCMC code was used to establish initial guesses for each galaxy of recessional velocity (redshift) and $\sigma$ (this routine uses course measurements of redshift taken from SIMBAD \citep{Wenger2000} for its own initial guess and $\sigma = 200 \mathrm{km s^{-1}}$).

		Ionized gas detection was treated in a similar way to \citet{Sarzi2005}: only if an amplitude to noise ratio of greater than 4 was observed for the given bin would the detection be included. H$_\mathrm{\delta}$ is detected in 3/10 galaxies and [NI] is redshifted out of the spectral range is 3/10 galaxies.



\section{Kinematics}
	\label{sec:kine}
	The kinematics of the sample are classified according to the Regular-Rotator/Non Regular-Rotator (RR/NRR) regime given in \citet{Krajnovic2011}, Fast/Slow Rotator (FR/SR) regime given in \citet{Cappellari2016} (originally defined by \citet{Emsellem2011}, but later refined by \citet{Cappellari2016}). Beyond this attempts have been made to use the kinematic features as defined in \citet{Krajnovic2011}, however the quality of the data has meant that many have had to be classified by eye as the artifacts from the VIMOS quadrants confuse any ellipse fitting methods. 

	The following briefly describe the observations and results of each of the sample. %It is also worth noting that many maps have been clipped in the color axis, to allow the more detailed structures to not be overwhelmed by the extremes. 

	\textbf{IC 1459} is a known to contain a KDC. This is clearly seen in our velocity map (Fig. \ref{fig:stellar_vel}). It is also known to have ionized gas counter rotating to the decoupled core. This is again seen by comparing Fig. \ref{fig:stellar_vel} with Fig. \ref{fig:OIII_vel}, \ref{fig:Hbeta_vel} and possibly \ref{fig:NI_vel}. Our field of view is not large enough to investigate if the gas aligns with the outer galaxy.

	\textbf{IC 1531} seems to contain a KT. This galaxy has a very limited detection of ionized gas concentrated in the center (with the exception of [NI] (Fig. \ref{fig:NI_eqW}), which is more dispersed).

	\textbf{IC 4296} appears to have KT, though this may be a quadrant feature. There is potentially 2 peaks in all of the gas intensity maps, one at the center of the galaxy and one to the south-east which is not seen in the image (total flux/collapsed cube).

	\textbf{NGC 0612} has a large dust lane to the east of the apparent center of the galaxy perpendicular to the axis of rotation. The dust lane is also seen as a lower velocity dispersion (Fig \ref{fig:stellar_sigma}). Dust lanes generally imply a disky galaxy: indeed this seems to be the case here as the dust lane is aligned with the plane of the disk. The kinematic maps show NF. The dust lane also contains large amounts of gas. 

	\textbf{NGC 1399}, the central galaxy of the Fornax Cluster \citep{Jordan2007}, is known to have kinematic twist (see MUSE map in \citet{Zieleniewski2017}), however this is not visible in our map. This is partly due to the twist being on a scale of about the VIMOS field of view and partly due to the quadrant features in our maps. We have used this classification none the less. Only H$_\mathrm{\beta}$ and H$_\mathrm{\delta}$ was detected, and only at very low levels. [NI] was redshifted out of the VIMOS spectral range.

	\textbf{NGC 3100} is has NF in the stellar kinematics, however there is significant amount of ionized gas, which seems to be split into two clouds. This is most obviously seen in Fig. \ref{fig:Hbeta_eqW}. The gas also seems to have a non-standard rotation, possibly linked to its spacial distribution. 

	\textbf{NGC 3557} is known to be FR with very high velocities, especially considering it's size, with NF. In our maps, there some significant quadrant effects. NGC 3557 also has a very dispersed, non-centrally concentrated H$_\mathrm{\beta}$ distribution. 

	\textbf{NGC 7075} appears to have NF, though with quite slow velocities. There is some H$_\mathrm{\beta}$ detected at the very center of the galaxy. 

	\textbf{PKS 0718-34} is a KDC, though S/N issues mean that as in IC 1459, the galaxy cannot be seen beyond the core. It has very little gas detected, though the often faint $H_\mathrm{\delta}$ line is detected. [NI] is redshifted out of the VIMOS spectral range.

	\textbf{ESO 443-G024} is consistent with a NR. It has a very dispersed H$_\mathrm{\beta}$ (similar to NGC 3557 (Fig. \ref{fig:Hbeta_eqW})).


	


\section{Discussion}
	\label{sec:discuss}

\section{Conclusion}
	\label{sec:conc}


%\FloatBarrier
\bibliographystyle{mnras}

\bibliography{bib}{}

\appendix

\section{Kinematic Maps}
	\label{sec:kinmaps}
    \subsection{Stellar maps}
        \label{subsec:stellarmaps}
        \begin{figure*}
            \centering
            \includegraphics[width=0.245\textwidth]{plots/ngc0612_stellar_img.png}
            \includegraphics[width=0.245\textwidth]{plots/ngc3557_stellar_img.png}
            \includegraphics[width=0.245\textwidth]{plots/ngc3100_stellar_img.png}
            \includegraphics[width=0.245\textwidth]{plots/ic1459_stellar_img.png}
            \includegraphics[width=0.245\textwidth]{plots/pks0718-34_stellar_img.png}
            \includegraphics[width=0.245\textwidth]{plots/ic4296_stellar_img.png}
            \includegraphics[width=0.245\textwidth]{plots/ngc7075_stellar_img.png}
            \includegraphics[width=0.245\textwidth]{plots/ic1531_stellar_img.png}
            \includegraphics[width=0.245\textwidth]{plots/ngc1399_stellar_img.png}
            \includegraphics[width=0.245\textwidth]{plots/eso443-g024_stellar_img.png}
            \caption{ image for each galaxy in the sample.}
            \label{fig:stellar_img}
        \end{figure*}


        \begin{figure*}
            \centering
            \includegraphics[width=0.245\textwidth]{plots/ngc0612_stellar_vel.png}
            \includegraphics[width=0.245\textwidth]{plots/ngc3557_stellar_vel.png}
            \includegraphics[width=0.245\textwidth]{plots/ngc3100_stellar_vel.png}
            \includegraphics[width=0.245\textwidth]{plots/ic1459_stellar_vel.png}
            \includegraphics[width=0.245\textwidth]{plots/pks0718-34_stellar_vel.png}
            \includegraphics[width=0.245\textwidth]{plots/ic4296_stellar_vel.png}
            \includegraphics[width=0.245\textwidth]{plots/ngc7075_stellar_vel.png}
            \includegraphics[width=0.245\textwidth]{plots/ic1531_stellar_vel.png}
            \includegraphics[width=0.245\textwidth]{plots/ngc1399_stellar_vel.png}
            \includegraphics[width=0.245\textwidth]{plots/eso443-g024_stellar_vel.png}
            \caption{ velocity map for each galaxy in the sample.}
            \label{fig:stellar_vel}
        \end{figure*}


        \begin{figure*}
            \centering
            \includegraphics[width=0.245\textwidth]{plots/ngc0612_stellar_sigma.png}
            \includegraphics[width=0.245\textwidth]{plots/ngc3557_stellar_sigma.png}
            \includegraphics[width=0.245\textwidth]{plots/ngc3100_stellar_sigma.png}
            \includegraphics[width=0.245\textwidth]{plots/ic1459_stellar_sigma.png}
            \includegraphics[width=0.245\textwidth]{plots/pks0718-34_stellar_sigma.png}
            \includegraphics[width=0.245\textwidth]{plots/ic4296_stellar_sigma.png}
            \includegraphics[width=0.245\textwidth]{plots/ngc7075_stellar_sigma.png}
            \includegraphics[width=0.245\textwidth]{plots/ic1531_stellar_sigma.png}
            \includegraphics[width=0.245\textwidth]{plots/ngc1399_stellar_sigma.png}
            \includegraphics[width=0.245\textwidth]{plots/eso443-g024_stellar_sigma.png}
            \caption{ velocity dispersion ($\mathrm{\sigma}$) map for each galaxy in the sample.}
            \label{fig:stellar_sigma}
        \end{figure*}


        \begin{figure*}
            \centering
            \includegraphics[width=0.245\textwidth]{plots/ngc0612_stellar_h3.png}
            \includegraphics[width=0.245\textwidth]{plots/ngc3557_stellar_h3.png}
            \includegraphics[width=0.245\textwidth]{plots/ngc3100_stellar_h3.png}
            \includegraphics[width=0.245\textwidth]{plots/ic1459_stellar_h3.png}
            \includegraphics[width=0.245\textwidth]{plots/pks0718-34_stellar_h3.png}
            \includegraphics[width=0.245\textwidth]{plots/ic4296_stellar_h3.png}
            \includegraphics[width=0.245\textwidth]{plots/ngc7075_stellar_h3.png}
            \includegraphics[width=0.245\textwidth]{plots/ic1531_stellar_h3.png}
            \includegraphics[width=0.245\textwidth]{plots/ngc1399_stellar_h3.png}
            \includegraphics[width=0.245\textwidth]{plots/eso443-g024_stellar_h3.png}
            \caption{ third Guass-Hermite moment (h3) map for each galaxy in the sample.}
            \label{fig:stellar_h3}
        \end{figure*}


        \begin{figure*}
            \centering
            \includegraphics[width=0.245\textwidth]{plots/ngc0612_stellar_h4.png}
            \includegraphics[width=0.245\textwidth]{plots/ngc3557_stellar_h4.png}
            \includegraphics[width=0.245\textwidth]{plots/ngc3100_stellar_h4.png}
            \includegraphics[width=0.245\textwidth]{plots/ic1459_stellar_h4.png}
            \includegraphics[width=0.245\textwidth]{plots/pks0718-34_stellar_h4.png}
            \includegraphics[width=0.245\textwidth]{plots/ic4296_stellar_h4.png}
            \includegraphics[width=0.245\textwidth]{plots/ngc7075_stellar_h4.png}
            \includegraphics[width=0.245\textwidth]{plots/ic1531_stellar_h4.png}
            \includegraphics[width=0.245\textwidth]{plots/ngc1399_stellar_h4.png}
            \includegraphics[width=0.245\textwidth]{plots/eso443-g024_stellar_h4.png}
            \caption{ fourth Guass-Hermite moment (h4) map for each galaxy in the sample.}
            \label{fig:stellar_h4}
        \end{figure*}





    \subsection{OIII maps}
        \label{subsec:OIIImaps}
        \begin{figure*}
            \centering
            \includegraphics[width=0.245\textwidth]{plots/ngc0612_[OIII]5007d_img.png}
            \includegraphics[width=0.245\textwidth]{plots/ngc3557_[OIII]5007d_img.png}
            \includegraphics[width=0.245\textwidth]{plots/ngc3100_[OIII]5007d_img.png}
            \includegraphics[width=0.245\textwidth]{plots/ic1459_[OIII]5007d_img.png}
            \includegraphics[width=0.245\textwidth]{plots/pks0718-34_[OIII]5007d_img.png}
            \includegraphics[width=0.245\textwidth]{plots/ic4296_[OIII]5007d_img.png}
            \includegraphics[width=0.245\textwidth]{plots/ngc7075_[OIII]5007d_img.png}
            \includegraphics[width=0.245\textwidth]{plots/ic1531_[OIII]5007d_img.png}
            \includegraphics[width=0.245\textwidth]{plots/ngc1399_[OIII]5007d_img.png}
            \includegraphics[width=0.245\textwidth]{plots/eso443-g024_[OIII]5007d_img.png}
            \caption{[OIII] image for each galaxy in the sample.}
            \label{fig:OIII_img}
        \end{figure*}


        \begin{figure*}
            \centering
            \includegraphics[width=0.245\textwidth]{plots/ngc0612_[OIII]5007d_eqW.png}
            \includegraphics[width=0.245\textwidth]{plots/ngc3557_[OIII]5007d_eqW.png}
            \includegraphics[width=0.245\textwidth]{plots/ngc3100_[OIII]5007d_eqW.png}
            \includegraphics[width=0.245\textwidth]{plots/ic1459_[OIII]5007d_eqW.png}
            \includegraphics[width=0.245\textwidth]{plots/pks0718-34_[OIII]5007d_eqW.png}
            \includegraphics[width=0.245\textwidth]{plots/ic4296_[OIII]5007d_eqW.png}
            \includegraphics[width=0.245\textwidth]{plots/ngc7075_[OIII]5007d_eqW.png}
            \includegraphics[width=0.245\textwidth]{plots/ic1531_[OIII]5007d_eqW.png}
            \includegraphics[width=0.245\textwidth]{plots/ngc1399_[OIII]5007d_eqW.png}
            \includegraphics[width=0.245\textwidth]{plots/eso443-g024_[OIII]5007d_eqW.png}
            \caption{[OIII] equivelent width for each galaxy in the sample.}
            \label{fig:OIII_eqW}
        \end{figure*}


        \begin{figure*}
            \centering
            \includegraphics[width=0.245\textwidth]{plots/ngc0612_[OIII]5007d_vel.png}
            \includegraphics[width=0.245\textwidth]{plots/ngc3557_[OIII]5007d_vel.png}
            \includegraphics[width=0.245\textwidth]{plots/ngc3100_[OIII]5007d_vel.png}
            \includegraphics[width=0.245\textwidth]{plots/ic1459_[OIII]5007d_vel.png}
            \includegraphics[width=0.245\textwidth]{plots/pks0718-34_[OIII]5007d_vel.png}
            \includegraphics[width=0.245\textwidth]{plots/ic4296_[OIII]5007d_vel.png}
            \includegraphics[width=0.245\textwidth]{plots/ngc7075_[OIII]5007d_vel.png}
            \includegraphics[width=0.245\textwidth]{plots/ic1531_[OIII]5007d_vel.png}
            \includegraphics[width=0.245\textwidth]{plots/ngc1399_[OIII]5007d_vel.png}
            \includegraphics[width=0.245\textwidth]{plots/eso443-g024_[OIII]5007d_vel.png}
            \caption{[OIII] velocity map for each galaxy in the sample.}
            \label{fig:OIII_vel}
        \end{figure*}


        \begin{figure*}
            \centering
            \includegraphics[width=0.245\textwidth]{plots/ngc0612_[OIII]5007d_sigma.png}
            \includegraphics[width=0.245\textwidth]{plots/ngc3557_[OIII]5007d_sigma.png}
            \includegraphics[width=0.245\textwidth]{plots/ngc3100_[OIII]5007d_sigma.png}
            \includegraphics[width=0.245\textwidth]{plots/ic1459_[OIII]5007d_sigma.png}
            \includegraphics[width=0.245\textwidth]{plots/pks0718-34_[OIII]5007d_sigma.png}
            \includegraphics[width=0.245\textwidth]{plots/ic4296_[OIII]5007d_sigma.png}
            \includegraphics[width=0.245\textwidth]{plots/ngc7075_[OIII]5007d_sigma.png}
            \includegraphics[width=0.245\textwidth]{plots/ic1531_[OIII]5007d_sigma.png}
            \includegraphics[width=0.245\textwidth]{plots/ngc1399_[OIII]5007d_sigma.png}
            \includegraphics[width=0.245\textwidth]{plots/eso443-g024_[OIII]5007d_sigma.png}
            \caption{[OIII] velocity dispersion ($\mathrm{\sigma}$) map for each galaxy in the sample.}
            \label{fig:OIII_sigma}
        \end{figure*}


        \begin{figure*}
            \centering
            \includegraphics[width=0.245\textwidth]{plots/ngc0612_[OIII]5007d_h3.png}
            \includegraphics[width=0.245\textwidth]{plots/ngc3557_[OIII]5007d_h3.png}
            \includegraphics[width=0.245\textwidth]{plots/ngc3100_[OIII]5007d_h3.png}
            \includegraphics[width=0.245\textwidth]{plots/ic1459_[OIII]5007d_h3.png}
            \includegraphics[width=0.245\textwidth]{plots/pks0718-34_[OIII]5007d_h3.png}
            \includegraphics[width=0.245\textwidth]{plots/ic4296_[OIII]5007d_h3.png}
            \includegraphics[width=0.245\textwidth]{plots/ngc7075_[OIII]5007d_h3.png}
            \includegraphics[width=0.245\textwidth]{plots/ic1531_[OIII]5007d_h3.png}
            \includegraphics[width=0.245\textwidth]{plots/ngc1399_[OIII]5007d_h3.png}
            \includegraphics[width=0.245\textwidth]{plots/eso443-g024_[OIII]5007d_h3.png}
            \caption{[OIII] third Guass-Hermite moment (h3) map for each galaxy in the sample.}
            \label{fig:OIII_h3}
        \end{figure*}


        \begin{figure*}
            \centering
            \includegraphics[width=0.245\textwidth]{plots/ngc0612_[OIII]5007d_h4.png}
            \includegraphics[width=0.245\textwidth]{plots/ngc3557_[OIII]5007d_h4.png}
            \includegraphics[width=0.245\textwidth]{plots/ngc3100_[OIII]5007d_h4.png}
            \includegraphics[width=0.245\textwidth]{plots/ic1459_[OIII]5007d_h4.png}
            \includegraphics[width=0.245\textwidth]{plots/pks0718-34_[OIII]5007d_h4.png}
            \includegraphics[width=0.245\textwidth]{plots/ic4296_[OIII]5007d_h4.png}
            \includegraphics[width=0.245\textwidth]{plots/ngc7075_[OIII]5007d_h4.png}
            \includegraphics[width=0.245\textwidth]{plots/ic1531_[OIII]5007d_h4.png}
            \includegraphics[width=0.245\textwidth]{plots/ngc1399_[OIII]5007d_h4.png}
            \includegraphics[width=0.245\textwidth]{plots/eso443-g024_[OIII]5007d_h4.png}
            \caption{[OIII] fourth Guass-Hermite moment (h4) map for each galaxy in the sample.}
            \label{fig:OIII_h4}
        \end{figure*}





    \subsection{Hbeta maps}
        \label{subsec:Hbetamaps}
        \begin{figure*}
            \centering
            \includegraphics[width=0.245\textwidth]{plots/ngc0612_Hbeta_img.png}
            \includegraphics[width=0.245\textwidth]{plots/ngc3557_Hbeta_img.png}
            \includegraphics[width=0.245\textwidth]{plots/ngc3100_Hbeta_img.png}
            \includegraphics[width=0.245\textwidth]{plots/ic1459_Hbeta_img.png}
            \includegraphics[width=0.245\textwidth]{plots/pks0718-34_Hbeta_img.png}
            \includegraphics[width=0.245\textwidth]{plots/ic4296_Hbeta_img.png}
            \includegraphics[width=0.245\textwidth]{plots/ngc7075_Hbeta_img.png}
            \includegraphics[width=0.245\textwidth]{plots/ic1531_Hbeta_img.png}
            \includegraphics[width=0.245\textwidth]{plots/ngc1399_Hbeta_img.png}
            \includegraphics[width=0.245\textwidth]{plots/eso443-g024_Hbeta_img.png}
            \caption{H$_\mathrm{\beta}$ image for each galaxy in the sample.}
            \label{fig:Hbeta_img}
        \end{figure*}


        \begin{figure*}
            \centering
            \includegraphics[width=0.245\textwidth]{plots/ngc0612_Hbeta_eqW.png}
            \includegraphics[width=0.245\textwidth]{plots/ngc3557_Hbeta_eqW.png}
            \includegraphics[width=0.245\textwidth]{plots/ngc3100_Hbeta_eqW.png}
            \includegraphics[width=0.245\textwidth]{plots/ic1459_Hbeta_eqW.png}
            \includegraphics[width=0.245\textwidth]{plots/pks0718-34_Hbeta_eqW.png}
            \includegraphics[width=0.245\textwidth]{plots/ic4296_Hbeta_eqW.png}
            \includegraphics[width=0.245\textwidth]{plots/ngc7075_Hbeta_eqW.png}
            \includegraphics[width=0.245\textwidth]{plots/ic1531_Hbeta_eqW.png}
            \includegraphics[width=0.245\textwidth]{plots/ngc1399_Hbeta_eqW.png}
            \includegraphics[width=0.245\textwidth]{plots/eso443-g024_Hbeta_eqW.png}
            \caption{H$_\mathrm{\beta}$ equivelent width for each galaxy in the sample.}
            \label{fig:Hbeta_eqW}
        \end{figure*}


        \begin{figure*}
            \centering
            \includegraphics[width=0.245\textwidth]{plots/ngc0612_Hbeta_vel.png}
            \includegraphics[width=0.245\textwidth]{plots/ngc3557_Hbeta_vel.png}
            \includegraphics[width=0.245\textwidth]{plots/ngc3100_Hbeta_vel.png}
            \includegraphics[width=0.245\textwidth]{plots/ic1459_Hbeta_vel.png}
            \includegraphics[width=0.245\textwidth]{plots/pks0718-34_Hbeta_vel.png}
            \includegraphics[width=0.245\textwidth]{plots/ic4296_Hbeta_vel.png}
            \includegraphics[width=0.245\textwidth]{plots/ngc7075_Hbeta_vel.png}
            \includegraphics[width=0.245\textwidth]{plots/ic1531_Hbeta_vel.png}
            \includegraphics[width=0.245\textwidth]{plots/ngc1399_Hbeta_vel.png}
            \includegraphics[width=0.245\textwidth]{plots/eso443-g024_Hbeta_vel.png}
            \caption{H$_\mathrm{\beta}$ velocity map for each galaxy in the sample.}
            \label{fig:Hbeta_vel}
        \end{figure*}


        \begin{figure*}
            \centering
            \includegraphics[width=0.245\textwidth]{plots/ngc0612_Hbeta_sigma.png}
            \includegraphics[width=0.245\textwidth]{plots/ngc3557_Hbeta_sigma.png}
            \includegraphics[width=0.245\textwidth]{plots/ngc3100_Hbeta_sigma.png}
            \includegraphics[width=0.245\textwidth]{plots/ic1459_Hbeta_sigma.png}
            \includegraphics[width=0.245\textwidth]{plots/pks0718-34_Hbeta_sigma.png}
            \includegraphics[width=0.245\textwidth]{plots/ic4296_Hbeta_sigma.png}
            \includegraphics[width=0.245\textwidth]{plots/ngc7075_Hbeta_sigma.png}
            \includegraphics[width=0.245\textwidth]{plots/ic1531_Hbeta_sigma.png}
            \includegraphics[width=0.245\textwidth]{plots/ngc1399_Hbeta_sigma.png}
            \includegraphics[width=0.245\textwidth]{plots/eso443-g024_Hbeta_sigma.png}
            \caption{H$_\mathrm{\beta}$ velocity dispersion ($\mathrm{\sigma}$) map for each galaxy in the sample.}
            \label{fig:Hbeta_sigma}
        \end{figure*}


        \begin{figure*}
            \centering
            \includegraphics[width=0.245\textwidth]{plots/ngc0612_Hbeta_h3.png}
            \includegraphics[width=0.245\textwidth]{plots/ngc3557_Hbeta_h3.png}
            \includegraphics[width=0.245\textwidth]{plots/ngc3100_Hbeta_h3.png}
            \includegraphics[width=0.245\textwidth]{plots/ic1459_Hbeta_h3.png}
            \includegraphics[width=0.245\textwidth]{plots/pks0718-34_Hbeta_h3.png}
            \includegraphics[width=0.245\textwidth]{plots/ic4296_Hbeta_h3.png}
            \includegraphics[width=0.245\textwidth]{plots/ngc7075_Hbeta_h3.png}
            \includegraphics[width=0.245\textwidth]{plots/ic1531_Hbeta_h3.png}
            \includegraphics[width=0.245\textwidth]{plots/ngc1399_Hbeta_h3.png}
            \includegraphics[width=0.245\textwidth]{plots/eso443-g024_Hbeta_h3.png}
            \caption{H$_\mathrm{\beta}$ third Guass-Hermite moment (h3) map for each galaxy in the sample.}
            \label{fig:Hbeta_h3}
        \end{figure*}


        \begin{figure*}
            \centering
            \includegraphics[width=0.245\textwidth]{plots/ngc0612_Hbeta_h4.png}
            \includegraphics[width=0.245\textwidth]{plots/ngc3557_Hbeta_h4.png}
            \includegraphics[width=0.245\textwidth]{plots/ngc3100_Hbeta_h4.png}
            \includegraphics[width=0.245\textwidth]{plots/ic1459_Hbeta_h4.png}
            \includegraphics[width=0.245\textwidth]{plots/pks0718-34_Hbeta_h4.png}
            \includegraphics[width=0.245\textwidth]{plots/ic4296_Hbeta_h4.png}
            \includegraphics[width=0.245\textwidth]{plots/ngc7075_Hbeta_h4.png}
            \includegraphics[width=0.245\textwidth]{plots/ic1531_Hbeta_h4.png}
            \includegraphics[width=0.245\textwidth]{plots/ngc1399_Hbeta_h4.png}
            \includegraphics[width=0.245\textwidth]{plots/eso443-g024_Hbeta_h4.png}
            \caption{H$_\mathrm{\beta}$ fourth Guass-Hermite moment (h4) map for each galaxy in the sample.}
            \label{fig:Hbeta_h4}
        \end{figure*}





    \subsection{Hgamma maps}
        \label{subsec:Hgammamaps}
        \begin{figure*}
            \centering
            \includegraphics[width=0.245\textwidth]{plots/ngc0612_Hgamma_img.png}
            \includegraphics[width=0.245\textwidth]{plots/ngc3557_Hgamma_img.png}
            \includegraphics[width=0.245\textwidth]{plots/ngc3100_Hgamma_img.png}
            \includegraphics[width=0.245\textwidth]{plots/ic1459_Hgamma_img.png}
            \includegraphics[width=0.245\textwidth]{plots/pks0718-34_Hgamma_img.png}
            \includegraphics[width=0.245\textwidth]{plots/ic4296_Hgamma_img.png}
            \includegraphics[width=0.245\textwidth]{plots/ngc7075_Hgamma_img.png}
            \includegraphics[width=0.245\textwidth]{plots/ic1531_Hgamma_img.png}
            \includegraphics[width=0.245\textwidth]{plots/ngc1399_Hgamma_img.png}
            \includegraphics[width=0.245\textwidth]{plots/eso443-g024_Hgamma_img.png}
            \caption{H$_\mathrm{\gamma}$ image for each galaxy in the sample.}
            \label{fig:Hgamma_img}
        \end{figure*}


        \begin{figure*}
            \centering
            \includegraphics[width=0.245\textwidth]{plots/ngc0612_Hgamma_eqW.png}
            \includegraphics[width=0.245\textwidth]{plots/ngc3557_Hgamma_eqW.png}
            \includegraphics[width=0.245\textwidth]{plots/ngc3100_Hgamma_eqW.png}
            \includegraphics[width=0.245\textwidth]{plots/ic1459_Hgamma_eqW.png}
            \includegraphics[width=0.245\textwidth]{plots/pks0718-34_Hgamma_eqW.png}
            \includegraphics[width=0.245\textwidth]{plots/ic4296_Hgamma_eqW.png}
            \includegraphics[width=0.245\textwidth]{plots/ngc7075_Hgamma_eqW.png}
            \includegraphics[width=0.245\textwidth]{plots/ic1531_Hgamma_eqW.png}
            \includegraphics[width=0.245\textwidth]{plots/ngc1399_Hgamma_eqW.png}
            \includegraphics[width=0.245\textwidth]{plots/eso443-g024_Hgamma_eqW.png}
            \caption{H$_\mathrm{\gamma}$ equivelent width for each galaxy in the sample.}
            \label{fig:Hgamma_eqW}
        \end{figure*}


        \begin{figure*}
            \centering
            \includegraphics[width=0.245\textwidth]{plots/ngc0612_Hgamma_vel.png}
            \includegraphics[width=0.245\textwidth]{plots/ngc3557_Hgamma_vel.png}
            \includegraphics[width=0.245\textwidth]{plots/ngc3100_Hgamma_vel.png}
            \includegraphics[width=0.245\textwidth]{plots/ic1459_Hgamma_vel.png}
            \includegraphics[width=0.245\textwidth]{plots/pks0718-34_Hgamma_vel.png}
            \includegraphics[width=0.245\textwidth]{plots/ic4296_Hgamma_vel.png}
            \includegraphics[width=0.245\textwidth]{plots/ngc7075_Hgamma_vel.png}
            \includegraphics[width=0.245\textwidth]{plots/ic1531_Hgamma_vel.png}
            \includegraphics[width=0.245\textwidth]{plots/ngc1399_Hgamma_vel.png}
            \includegraphics[width=0.245\textwidth]{plots/eso443-g024_Hgamma_vel.png}
            \caption{H$_\mathrm{\gamma}$ velocity map for each galaxy in the sample.}
            \label{fig:Hgamma_vel}
        \end{figure*}


        \begin{figure*}
            \centering
            \includegraphics[width=0.245\textwidth]{plots/ngc0612_Hgamma_sigma.png}
            \includegraphics[width=0.245\textwidth]{plots/ngc3557_Hgamma_sigma.png}
            \includegraphics[width=0.245\textwidth]{plots/ngc3100_Hgamma_sigma.png}
            \includegraphics[width=0.245\textwidth]{plots/ic1459_Hgamma_sigma.png}
            \includegraphics[width=0.245\textwidth]{plots/pks0718-34_Hgamma_sigma.png}
            \includegraphics[width=0.245\textwidth]{plots/ic4296_Hgamma_sigma.png}
            \includegraphics[width=0.245\textwidth]{plots/ngc7075_Hgamma_sigma.png}
            \includegraphics[width=0.245\textwidth]{plots/ic1531_Hgamma_sigma.png}
            \includegraphics[width=0.245\textwidth]{plots/ngc1399_Hgamma_sigma.png}
            \includegraphics[width=0.245\textwidth]{plots/eso443-g024_Hgamma_sigma.png}
            \caption{H$_\mathrm{\gamma}$ velocity dispersion ($\mathrm{\sigma}$) map for each galaxy in the sample.}
            \label{fig:Hgamma_sigma}
        \end{figure*}


        \begin{figure*}
            \centering
            \includegraphics[width=0.245\textwidth]{plots/ngc0612_Hgamma_h3.png}
            \includegraphics[width=0.245\textwidth]{plots/ngc3557_Hgamma_h3.png}
            \includegraphics[width=0.245\textwidth]{plots/ngc3100_Hgamma_h3.png}
            \includegraphics[width=0.245\textwidth]{plots/ic1459_Hgamma_h3.png}
            \includegraphics[width=0.245\textwidth]{plots/pks0718-34_Hgamma_h3.png}
            \includegraphics[width=0.245\textwidth]{plots/ic4296_Hgamma_h3.png}
            \includegraphics[width=0.245\textwidth]{plots/ngc7075_Hgamma_h3.png}
            \includegraphics[width=0.245\textwidth]{plots/ic1531_Hgamma_h3.png}
            \includegraphics[width=0.245\textwidth]{plots/ngc1399_Hgamma_h3.png}
            \includegraphics[width=0.245\textwidth]{plots/eso443-g024_Hgamma_h3.png}
            \caption{H$_\mathrm{\gamma}$ third Guass-Hermite moment (h3) map for each galaxy in the sample.}
            \label{fig:Hgamma_h3}
        \end{figure*}


        \begin{figure*}
            \centering
            \includegraphics[width=0.245\textwidth]{plots/ngc0612_Hgamma_h4.png}
            \includegraphics[width=0.245\textwidth]{plots/ngc3557_Hgamma_h4.png}
            \includegraphics[width=0.245\textwidth]{plots/ngc3100_Hgamma_h4.png}
            \includegraphics[width=0.245\textwidth]{plots/ic1459_Hgamma_h4.png}
            \includegraphics[width=0.245\textwidth]{plots/pks0718-34_Hgamma_h4.png}
            \includegraphics[width=0.245\textwidth]{plots/ic4296_Hgamma_h4.png}
            \includegraphics[width=0.245\textwidth]{plots/ngc7075_Hgamma_h4.png}
            \includegraphics[width=0.245\textwidth]{plots/ic1531_Hgamma_h4.png}
            \includegraphics[width=0.245\textwidth]{plots/ngc1399_Hgamma_h4.png}
            \includegraphics[width=0.245\textwidth]{plots/eso443-g024_Hgamma_h4.png}
            \caption{H$_\mathrm{\gamma}$ fourth Guass-Hermite moment (h4) map for each galaxy in the sample.}
            \label{fig:Hgamma_h4}
        \end{figure*}





    \subsection{Hdelta maps}
        \label{subsec:Hdeltamaps}
        \begin{figure*}
            \centering
            \includegraphics[width=0.245\textwidth]{plots/ngc0612_Hdelta_img.png}
            \includegraphics[width=0.245\textwidth]{plots/ngc3557_Hdelta_img.png}
            \includegraphics[width=0.245\textwidth]{plots/ngc3100_Hdelta_img.png}
            \includegraphics[width=0.245\textwidth]{plots/ic1459_Hdelta_img.png}
            \includegraphics[width=0.245\textwidth]{plots/pks0718-34_Hdelta_img.png}
            \includegraphics[width=0.245\textwidth]{plots/ic4296_Hdelta_img.png}
            \includegraphics[width=0.245\textwidth]{plots/ngc7075_Hdelta_img.png}
            \includegraphics[width=0.245\textwidth]{plots/ic1531_Hdelta_img.png}
            \includegraphics[width=0.245\textwidth]{plots/ngc1399_Hdelta_img.png}
            \includegraphics[width=0.245\textwidth]{plots/eso443-g024_Hdelta_img.png}
            \caption{H$_\mathrm{\gamma}$ image for each galaxy in the sample.}
            \label{fig:Hdelta_img}
        \end{figure*}


        \begin{figure*}
            \centering
            \includegraphics[width=0.245\textwidth]{plots/ngc0612_Hdelta_eqW.png}
            \includegraphics[width=0.245\textwidth]{plots/ngc3557_Hdelta_eqW.png}
            \includegraphics[width=0.245\textwidth]{plots/ngc3100_Hdelta_eqW.png}
            \includegraphics[width=0.245\textwidth]{plots/ic1459_Hdelta_eqW.png}
            \includegraphics[width=0.245\textwidth]{plots/pks0718-34_Hdelta_eqW.png}
            \includegraphics[width=0.245\textwidth]{plots/ic4296_Hdelta_eqW.png}
            \includegraphics[width=0.245\textwidth]{plots/ngc7075_Hdelta_eqW.png}
            \includegraphics[width=0.245\textwidth]{plots/ic1531_Hdelta_eqW.png}
            \includegraphics[width=0.245\textwidth]{plots/ngc1399_Hdelta_eqW.png}
            \includegraphics[width=0.245\textwidth]{plots/eso443-g024_Hdelta_eqW.png}
            \caption{H$_\mathrm{\gamma}$ equivelent width for each galaxy in the sample.}
            \label{fig:Hdelta_eqW}
        \end{figure*}


        \begin{figure*}
            \centering
            \includegraphics[width=0.245\textwidth]{plots/ngc0612_Hdelta_vel.png}
            \includegraphics[width=0.245\textwidth]{plots/ngc3557_Hdelta_vel.png}
            \includegraphics[width=0.245\textwidth]{plots/ngc3100_Hdelta_vel.png}
            \includegraphics[width=0.245\textwidth]{plots/ic1459_Hdelta_vel.png}
            \includegraphics[width=0.245\textwidth]{plots/pks0718-34_Hdelta_vel.png}
            \includegraphics[width=0.245\textwidth]{plots/ic4296_Hdelta_vel.png}
            \includegraphics[width=0.245\textwidth]{plots/ngc7075_Hdelta_vel.png}
            \includegraphics[width=0.245\textwidth]{plots/ic1531_Hdelta_vel.png}
            \includegraphics[width=0.245\textwidth]{plots/ngc1399_Hdelta_vel.png}
            \includegraphics[width=0.245\textwidth]{plots/eso443-g024_Hdelta_vel.png}
            \caption{H$_\mathrm{\gamma}$ velocity map for each galaxy in the sample.}
            \label{fig:Hdelta_vel}
        \end{figure*}


        \begin{figure*}
            \centering
            \includegraphics[width=0.245\textwidth]{plots/ngc0612_Hdelta_sigma.png}
            \includegraphics[width=0.245\textwidth]{plots/ngc3557_Hdelta_sigma.png}
            \includegraphics[width=0.245\textwidth]{plots/ngc3100_Hdelta_sigma.png}
            \includegraphics[width=0.245\textwidth]{plots/ic1459_Hdelta_sigma.png}
            \includegraphics[width=0.245\textwidth]{plots/pks0718-34_Hdelta_sigma.png}
            \includegraphics[width=0.245\textwidth]{plots/ic4296_Hdelta_sigma.png}
            \includegraphics[width=0.245\textwidth]{plots/ngc7075_Hdelta_sigma.png}
            \includegraphics[width=0.245\textwidth]{plots/ic1531_Hdelta_sigma.png}
            \includegraphics[width=0.245\textwidth]{plots/ngc1399_Hdelta_sigma.png}
            \includegraphics[width=0.245\textwidth]{plots/eso443-g024_Hdelta_sigma.png}
            \caption{H$_\mathrm{\gamma}$ velocity dispersion ($\mathrm{\sigma}$) map for each galaxy in the sample.}
            \label{fig:Hdelta_sigma}
        \end{figure*}


        \begin{figure*}
            \centering
            \includegraphics[width=0.245\textwidth]{plots/ngc0612_Hdelta_h3.png}
            \includegraphics[width=0.245\textwidth]{plots/ngc3557_Hdelta_h3.png}
            \includegraphics[width=0.245\textwidth]{plots/ngc3100_Hdelta_h3.png}
            \includegraphics[width=0.245\textwidth]{plots/ic1459_Hdelta_h3.png}
            \includegraphics[width=0.245\textwidth]{plots/pks0718-34_Hdelta_h3.png}
            \includegraphics[width=0.245\textwidth]{plots/ic4296_Hdelta_h3.png}
            \includegraphics[width=0.245\textwidth]{plots/ngc7075_Hdelta_h3.png}
            \includegraphics[width=0.245\textwidth]{plots/ic1531_Hdelta_h3.png}
            \includegraphics[width=0.245\textwidth]{plots/ngc1399_Hdelta_h3.png}
            \includegraphics[width=0.245\textwidth]{plots/eso443-g024_Hdelta_h3.png}
            \caption{H$_\mathrm{\gamma}$ third Guass-Hermite moment (h3) map for each galaxy in the sample.}
            \label{fig:Hdelta_h3}
        \end{figure*}


        \begin{figure*}
            \centering
            \includegraphics[width=0.245\textwidth]{plots/ngc0612_Hdelta_h4.png}
            \includegraphics[width=0.245\textwidth]{plots/ngc3557_Hdelta_h4.png}
            \includegraphics[width=0.245\textwidth]{plots/ngc3100_Hdelta_h4.png}
            \includegraphics[width=0.245\textwidth]{plots/ic1459_Hdelta_h4.png}
            \includegraphics[width=0.245\textwidth]{plots/pks0718-34_Hdelta_h4.png}
            \includegraphics[width=0.245\textwidth]{plots/ic4296_Hdelta_h4.png}
            \includegraphics[width=0.245\textwidth]{plots/ngc7075_Hdelta_h4.png}
            \includegraphics[width=0.245\textwidth]{plots/ic1531_Hdelta_h4.png}
            \includegraphics[width=0.245\textwidth]{plots/ngc1399_Hdelta_h4.png}
            \includegraphics[width=0.245\textwidth]{plots/eso443-g024_Hdelta_h4.png}
            \caption{H$_\mathrm{\gamma}$ fourth Guass-Hermite moment (h4) map for each galaxy in the sample.}
            \label{fig:Hdelta_h4}
        \end{figure*}





    \subsection{NI maps}
        \label{subsec:NImaps}
        \begin{figure*}
            \centering
            \includegraphics[width=0.245\textwidth]{plots/ngc0612_[NI]d_img.png}
            \includegraphics[width=0.245\textwidth]{plots/ngc3557_[NI]d_img.png}
            \includegraphics[width=0.245\textwidth]{plots/ngc3100_[NI]d_img.png}
            \includegraphics[width=0.245\textwidth]{plots/ic1459_[NI]d_img.png}
            \includegraphics[width=0.245\textwidth]{plots/pks0718-34_[NI]d_img.png}
            \includegraphics[width=0.245\textwidth]{plots/ic4296_[NI]d_img.png}
            \includegraphics[width=0.245\textwidth]{plots/ngc7075_[NI]d_img.png}
            \includegraphics[width=0.245\textwidth]{plots/ic1531_[NI]d_img.png}
            \includegraphics[width=0.245\textwidth]{plots/ngc1399_[NI]d_img.png}
            \includegraphics[width=0.245\textwidth]{plots/eso443-g024_[NI]d_img.png}
            \caption{[NI] image for each galaxy in the sample.}
            \label{fig:NI_img}
        \end{figure*}


        \begin{figure*}
            \centering
            \includegraphics[width=0.245\textwidth]{plots/ngc0612_[NI]d_eqW.png}
            \includegraphics[width=0.245\textwidth]{plots/ngc3557_[NI]d_eqW.png}
            \includegraphics[width=0.245\textwidth]{plots/ngc3100_[NI]d_eqW.png}
            \includegraphics[width=0.245\textwidth]{plots/ic1459_[NI]d_eqW.png}
            \includegraphics[width=0.245\textwidth]{plots/pks0718-34_[NI]d_eqW.png}
            \includegraphics[width=0.245\textwidth]{plots/ic4296_[NI]d_eqW.png}
            \includegraphics[width=0.245\textwidth]{plots/ngc7075_[NI]d_eqW.png}
            \includegraphics[width=0.245\textwidth]{plots/ic1531_[NI]d_eqW.png}
            \includegraphics[width=0.245\textwidth]{plots/ngc1399_[NI]d_eqW.png}
            \includegraphics[width=0.245\textwidth]{plots/eso443-g024_[NI]d_eqW.png}
            \caption{[NI] equivelent width for each galaxy in the sample.}
            \label{fig:NI_eqW}
        \end{figure*}


        \begin{figure*}
            \centering
            \includegraphics[width=0.245\textwidth]{plots/ngc0612_[NI]d_vel.png}
            \includegraphics[width=0.245\textwidth]{plots/ngc3557_[NI]d_vel.png}
            \includegraphics[width=0.245\textwidth]{plots/ngc3100_[NI]d_vel.png}
            \includegraphics[width=0.245\textwidth]{plots/ic1459_[NI]d_vel.png}
            \includegraphics[width=0.245\textwidth]{plots/pks0718-34_[NI]d_vel.png}
            \includegraphics[width=0.245\textwidth]{plots/ic4296_[NI]d_vel.png}
            \includegraphics[width=0.245\textwidth]{plots/ngc7075_[NI]d_vel.png}
            \includegraphics[width=0.245\textwidth]{plots/ic1531_[NI]d_vel.png}
            \includegraphics[width=0.245\textwidth]{plots/ngc1399_[NI]d_vel.png}
            \includegraphics[width=0.245\textwidth]{plots/eso443-g024_[NI]d_vel.png}
            \caption{[NI] velocity map for each galaxy in the sample.}
            \label{fig:NI_vel}
        \end{figure*}


        \begin{figure*}
            \centering
            \includegraphics[width=0.245\textwidth]{plots/ngc0612_[NI]d_sigma.png}
            \includegraphics[width=0.245\textwidth]{plots/ngc3557_[NI]d_sigma.png}
            \includegraphics[width=0.245\textwidth]{plots/ngc3100_[NI]d_sigma.png}
            \includegraphics[width=0.245\textwidth]{plots/ic1459_[NI]d_sigma.png}
            \includegraphics[width=0.245\textwidth]{plots/pks0718-34_[NI]d_sigma.png}
            \includegraphics[width=0.245\textwidth]{plots/ic4296_[NI]d_sigma.png}
            \includegraphics[width=0.245\textwidth]{plots/ngc7075_[NI]d_sigma.png}
            \includegraphics[width=0.245\textwidth]{plots/ic1531_[NI]d_sigma.png}
            \includegraphics[width=0.245\textwidth]{plots/ngc1399_[NI]d_sigma.png}
            \includegraphics[width=0.245\textwidth]{plots/eso443-g024_[NI]d_sigma.png}
            \caption{[NI] velocity dispersion ($\mathrm{\sigma}$) map for each galaxy in the sample.}
            \label{fig:NI_sigma}
        \end{figure*}


        \begin{figure*}
            \centering
            \includegraphics[width=0.245\textwidth]{plots/ngc0612_[NI]d_h3.png}
            \includegraphics[width=0.245\textwidth]{plots/ngc3557_[NI]d_h3.png}
            \includegraphics[width=0.245\textwidth]{plots/ngc3100_[NI]d_h3.png}
            \includegraphics[width=0.245\textwidth]{plots/ic1459_[NI]d_h3.png}
            \includegraphics[width=0.245\textwidth]{plots/pks0718-34_[NI]d_h3.png}
            \includegraphics[width=0.245\textwidth]{plots/ic4296_[NI]d_h3.png}
            \includegraphics[width=0.245\textwidth]{plots/ngc7075_[NI]d_h3.png}
            \includegraphics[width=0.245\textwidth]{plots/ic1531_[NI]d_h3.png}
            \includegraphics[width=0.245\textwidth]{plots/ngc1399_[NI]d_h3.png}
            \includegraphics[width=0.245\textwidth]{plots/eso443-g024_[NI]d_h3.png}
            \caption{[NI] third Guass-Hermite moment (h3) map for each galaxy in the sample.}
            \label{fig:NI_h3}
        \end{figure*}


        \begin{figure*}
            \centering
            \includegraphics[width=0.245\textwidth]{plots/ngc0612_[NI]d_h4.png}
            \includegraphics[width=0.245\textwidth]{plots/ngc3557_[NI]d_h4.png}
            \includegraphics[width=0.245\textwidth]{plots/ngc3100_[NI]d_h4.png}
            \includegraphics[width=0.245\textwidth]{plots/ic1459_[NI]d_h4.png}
            \includegraphics[width=0.245\textwidth]{plots/pks0718-34_[NI]d_h4.png}
            \includegraphics[width=0.245\textwidth]{plots/ic4296_[NI]d_h4.png}
            \includegraphics[width=0.245\textwidth]{plots/ngc7075_[NI]d_h4.png}
            \includegraphics[width=0.245\textwidth]{plots/ic1531_[NI]d_h4.png}
            \includegraphics[width=0.245\textwidth]{plots/ngc1399_[NI]d_h4.png}
            \includegraphics[width=0.245\textwidth]{plots/eso443-g024_[NI]d_h4.png}
            \caption{[NI] fourth Guass-Hermite moment (h4) map for each galaxy in the sample.}
            \label{fig:NI_h4}
        \end{figure*}






\end{document}